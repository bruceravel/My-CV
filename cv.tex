
\documentclass[11pt]{moderncv}
\moderncvtheme[purple]{casual}                 
%\moderncvtheme[purple]{classic}                 
                                             % optional argument are
                                             % 'blue' (default),
                                             % 'orange', 'red',
                                             % 'green', 'grey' and
                                             % 'roman' (for roman
                                             % fonts, instead of sans
                                             % serif fonts)
\usepackage[scale=0.8]{geometry}


\hyphenation{EXAFS}


% personal data
\firstname{Bruce}
\familyname{Ravel}
%\title{Bruce Ravel's curiculum vitae}
\address{Building 535A, Brookhaven National Laboratory}{Upton, NY 11973, USA}
\phone{1-631-344-3613}
%\mobile{1-312-636-3694}
\fax{1-631-344-5419}
\email{bravel@bnl.gov}
\homepage{http://xafs.org/BruceRavel}
\extrainfo{\href{http://github.com/bruceravel}{http://github.com/bruceravel}}
\photo[64pt]{BR_smiling.jpg}  % '64pt' is the height the picture must
                              % be resized to and 'picture' is the
                              % name of the picture file;
%\quote{Vertically integrated XAS method development}

% to show numerical labels in the bibliography; only useful if you make citations in your resume
\makeatletter
\renewcommand*{\bibliographyitemlabel}{\@biblabel{\arabic{enumiv}}}
\makeatother

% bibliography with mutiple entries
%\usepackage{multibib}
%\newcites{book,misc}{{Books},{Others}}

\begin{document}
\maketitle

\section{Employment History}
\cventry{2008--present}{Physicist}{National Institute of Standards and
Technology}{Gaithersburg MD}{USA}{
Synchrotron Methods Group at the National Synchrotron Light Source\\
Ceramics Division, Materials Measurement Laboratory
}
%
\cventry{2005--2008}{Assistant Physicist}{Argonne National
  Laboratory}{Argonne IL}{USA} {Molecular Environmental Science Group,
  Biosciences Division\newline{}Staff Scientist at The Materials
  Research Collaborative Access Team at the Advanced Photon Source}
%
\cventry{2001--2005}{Assistant Physicist}{Naval Research
  Laboratory}{Washington, DC}{USA}{ Chemistry Division\newline{}
  Spokesperson: NSLS Beamlines X11A, X11B, X23B }
%
\cventry{2000--2001}{Postdoctoral Fellow}{Naval Research
  Laboratory}{Washington, DC}{USA}{
Fellowship sponsored by The American Society of Electrical
Engineers\newline{}
Research Project: XAFS Studies of Half-Metallic Heusler Alloys
}
%
\cventry{1999--2000}{Visiting scientist}{Centre National de la
  Recherche Scientifique}{Grenoble}{France}{
  Research Project: Diffraction Anomalous Fine Structure Investigation of
  Superlattice Reflections in La$_{1/3}$Ca$_{2/3}$MnO$_3$\newline{}
  Staff member at European Synchrotron Radiation Source beamline BM02
}
%
\cventry{1997--1999}{Postdoctoral Fellow}{National Institute of
  Standards and Technology}{Gaithersburg, MD}{USA}{
  Fellowship sponsored by the National Research Council\newline{}
  Research Project: Diffraction Anomalous Fine Structure Investigation
  of the Ferroelectric Phase Transitions of BaTiO$_3$
}

\section{Education}
\cventry{1991--1997}{PhD, Physics}{University of Washington}{Seattle,
  WA}{}{\textbf{Thesis title}: Ferroelectric Phase Transitions in Oxide
  Perovskites Studied by XAFS\\
  \textbf{Advisors}: Dr. Edward A. Stern and Dr. John J. Rehr}  % arguments 3 to 6 can be left empty
\cventry{1989--1991}{MS, Physics}{University of Washington}{Seattle, WA}{}{}
\cventry{1985--1989}{BA, Physics}{Wesleyan University}{Middletown, CT}{}{}


\section{Research Interests}

\subsection{XAS Method Development}
\cvline{XAS Data Analysis Software}%
{I am a co-author of the widely used \textsc{ifeffit} analysis
  package.  My contributions include the graphical user interfaces
  \textsc{athena} (XAS Data Processing), \textsc{artemis} (EXAFS Data
  Analysis), and \textsc{hephaestus} (a periodic table for the X-ray
  absorption spectroscopist) as well as extensive documentation.
  \newline{} The publication introducing \textsc{athena},
  \textsc{artemis}, and \textsc{hephaestus} has over 1200 citations as
  of February, 2012, making it the highest cited article in the Journal of
  Synchrotron Radiation.}

\cvline{EXAFS analysis informed by MD and other theory}%
{Local structural theories -- such as molecular dynamics, Monte Carlo,
  and others -- show that common materials exhibit substantial local
  disorder that, in many cases, is poorly approximated by a simple
  cumulant expansion.  I have ongoing projects involving metallic
  nanoparticles and oxynitride dielectrics to explore how careful
  consideration of non-cumulant local disorder can be modeled in EXAFS
  analysis, including contributions of disorder to multiple scattering
  paths.}
%
\cvline{Integrative beamline technologies}%
{The advent of NSLS-II offers a rare opportunity to reconsider every
  aspect of beamline design and operation.  I am interested in finding
  common frameworks for handling all aspects of an XAS beamline,
  including controls, diagnostics, data acquisition, data processing,
  and data archiving.  The state of a beamline -- that is, the state
  of all controls, diagnostics, and acquisition instrumentation -- can
  be recorded continuously using scalable database technologies.  This
  complete record of state can be integrated into a data archiving
  strategy and inform automated or other sophisticated data processing
  strategies.}

\subsection{Beamline Development}
\cvline{NSLS X23A2}%
{I am the local contact at NIST's hard X-ray spectroscopy beamline
  with responsibilities that include user support and beamline
  maintenance.  My recent contributions to beamline operations include
  development of accurate deadtime corrections for silicon drift
  detectors and its integration into \textsc{athena} and the
  development of a novel four-channel ionization chamber for parallel
  measurement of four XAS spectra.}
%
\cvline{NSLS-II BMM}%
{With Joe Woicik, I am leading development of NIST's XAS and XRD
  beamline at NSLS-II.  The \textit{Beamline for Materials
    Measurement} will be on a 3-pole wiggler source.  With new optics,
  the performance of this new beamline will vastly exceed that of
  X23A2.  I am involved in all aspects of design and development and
  am interested in the development of novel control and data
  acquisition systems.}
%
\cvline{NSLS-II ISS}%
{I was the spokesperson for the \textit{Inner Shell Spectroscopy}
  beamline development proposal.  ISS was selected as one of the 6
  NEXT beamlines currently approaching the CD2 stage of development.
  I am working with BNL Photon Sciences staff to develop the
  preliminary design of this high flux, multipole wiggler beamline
  while leading the beamline advisory team to develop a scientific
  program including XAS on low concentration or low volume samples,
  high resolution fluorescence detection, X-ray emission spectroscopy,
  and X-ray energy loss spectroscopy.}



%\section{Teaching Experience}

\section{XAS Training Courses}
\cvlistitem{National Synchrotron Light Source, Nov.\ 8-10, 2012}%
\cvlistitem{State Univeristy of New York at Binghamton, Oct.\ 25-26, 2012}%
\cvlistitem{Karlsruhe Institute of Technology, Germany, Sep.\ 13-14, 2012}%
\cvlistitem{Synchrotron Light Research Institute, Khorat, Thailand,
  Aug.\ 13-15, 2012}%
\cvlistitem{Diamond Light Source, Nov.\ 14-16, 2011}%
\cvlistitem{National Synchrotron Light Source, Nov.\ 3-5, 2011}%
\cvlistitem{University of Ghent, Ghent Belgium, Jan.\ 12-14, 2011}%
\cvlistitem{National Synchrotron Light Source, Nov.\ 4-6, 2010}%
\cvlistitem{Johnson-Matthey Research Centre, Redding, England,
  Oct.\ 11-13, 2010}%
\cvlistitem{Synchrotron Light Research Institute, Khorat, Thailand,
  Aug.\ 4-6, 2010}%
\cvlistitem{University of Ghent, Ghent Belgium, Jan.\ 13-15, 2010}%
\cvlistitem{National Synchrotron Light Source, Oct.\ 22-24, 2009}%
\cvlistitem{Advanced Photon Source, July 6-10, 2009}%
\cvlistitem{Synchrotron Light Research Institute, Khorat, Thailand,
  May 27-29, 2009}%
\cvlistitem{National Synchrotron Light Source, Oct.\ 30 - Nov.\ 1, 2008}%
\cvlistitem{Swiss Light Source, Paul Scherrer Institute, Switzerland,
  Oct.\ 8-10, 2008}%
\cvlistitem{Canadian Light Source, Saskatoon, Saskatchewan, Canada,
  Aug.\ 20, 2008}%
\cvlistitem{Advanced Photon Source, Aug.\ 4-8, 2008}%
\cvlistitem{Advanced Photon Source, July 23-27, 2007}%
\cvlistitem{Physics Institute of the Polish Academy of Science,
  Warsaw, Nov.\ 13-15, 2006}%
\cvlistitem{Advanced Photon Source, July 26-28, 2006}%
\cvlistitem{Swiss Light Source, Paul Scherrer Institute, Switzerland,
  Feb.\ 20-21, 2006}%
\cvlistitem{Advanced Photon Source, July 26-29, 2005}%
\cvlistitem{Michigan State University, Jan.\ 20-21 , 2005}%
\cvlistitem{Canadian Light Source, Nov.\ 16-17, 2004}%
\cvlistitem{National Synchrotron Light Source, June 22-25, 2004}%
\cvlistitem{National Synchrotron Light Source, July 14-17, 2003}%
\cvlistitem{Alberta Synchrotron Institute, Nov.\ 12-14, 2002}%
\cvlistitem{National Synchrotron Light Source, Sep.\ 23-25, 2002}%
\cvlistitem{National Synchrotron Light Source, June 27-29, 2001}%
\cvlistitem{LNLS, Campinas, Brazil, May 7-9, 2001}%
\cvlistitem{University of Seville, Spain, May 24-26, 2000}%
\cvlistitem{CNRS, Grenoble France, May 17-19, 2000}%
\cvlistitem{University of Wuppertal, Germany, June 13-15, 2000}%
\cvlistitem{University of Leuven, Belgium, May 2-4, 2000}%
\cvlistitem{ESRF, Grenoble France, March 15-17, 2000}%
\cvlistitem{The Dow Chemical Company, Sep.\ 16-17, 1996}%

%HERCULES 2000, ``Introduction to FEFF Analysis of Absorption Spectra'', 29 March, 2000
%Irregular seminar series at NSLS, several dates, 2004
%Lecturer at the National School on Neutron and X-ray Scattering, 2005-2007

%\subsection{Other experience}

\section{Language Skills}
\cvlanguage{Spanish}{Proficient}{I first learned Spanish as a child living
  in Costa Rica and have maintained proficiency as an adult.}
\cvlanguage{French}{Conversational}{My French vocabulary is focused on
  a professional context due to my year working in Grenoble.}

\section{References}
\subsection{Supervisors}
\cvline{Daniel Fischer}%
{National Institute of Standards and Technology\newline
  Building 535A, Brookhaven National Laboratory, Upton, NY 11973, USA\newline
  \phonesymbol\ 1-631-344-5177\quad
  \href{mailto:dfischer@bnl.gov}{\emailsymbol\ \footnotesize\texttt{dfischer\char64bnl.gov}}}
\cvline{Kenneth Kemner}%
{Biosciences Division, Building 203\newline
  9700 South Cass Ave., Argonne National Laboratory, Argonne, IL 60439, USA\newline
  \phonesymbol\ 1-630-252-1163\quad
  \href{mailto:kemner@anl.gov}{\emailsymbol\ \footnotesize\texttt{kemner\char64anl.gov}}}
\cvline{Paul Natishan}%
{Chemistry Division, Code 6134\newline
  Naval Research Laboratory, Washington, DC 20375, USA\newline
  \phonesymbol\ 1-202-767-9255\quad
  \href{mailto:natishan@nrl.navy.mil}{\emailsymbol\ \footnotesize\texttt{natishan\char64nrl.navy.mil}}}

\subsection{Professional References}
\cvline{Joseph Woicik}%
{National Institute of Standards and Technology\newline
  Building 535A, Brookhaven National Laboratory, Upton, NY 11973, USA\newline
  \phonesymbol\ 1-631-344-4247\quad
  \href{mailto:woicik@bnl.gov}{\emailsymbol\ \footnotesize\texttt{woicik\char64bnl.gov}}}
\cvline{Matthew Newville}%
{Advanced Photon Source, Buidling 434\newline
  9700 South Cass Ave., Argonne National Laboratory, Argonne, IL 60439, USA\newline
  \phonesymbol\ 1-630-252-0431\quad
  \href{mailto:newville@cars.uchicago.edu}
  {\emailsymbol\ \footnotesize\texttt{newville\char64cars.uchicago.edu}}}
\cvline{John J. Rehr}%
{Department of Physics, Box 351560\newline
  University of Washington, Seattle, WA 98195-1560, USA\newline
  \phonesymbol\ 1-206-543-8593\quad
  \href{mailto:jjr@uw.edu}{\emailsymbol\ \footnotesize\texttt{jjr\char64uw.edu}}}

\section{Significant Publications}
\label{sec:highlights}

\cvline{\textsc{athena} and \textsc{artemis}}%
{\small Ravel, B. \& Newville, M. \textsc{athena} and
  \textsc{artemis}: data analysis for X-ray absorption spectroscopy
  using \textsc{ifeffit}, \emph{Journal of Synchrotron Radiation}
  \textbf{12}, 537--541 (2005)
  \href{http://dx.doi.org/10.1107/S0909049505012719}
  {\color{color2}\homepagesymbol~doi:10.1107/S0909049505012719}}

\cvline{\textsc{feff8}}%
{\small Ankudinov, A.~L., Ravel, B., Rehr, J.~J., Conradson, S.~D.,
  Real-space multiple-scattering calculation and interpretation of
  x-ray-absorption near-edge structure, \emph{Physical Review B}
  \textbf{58}, 7565--7576 (1998).
  \href{http://dx.doi.org/10.1103/PhysRevB.58.7565}
  {\color{color2}\homepagesymbol~doi:10.1103/PhysRevB.58.7565}}

\cvline{XAS Review}%
{\small Kelly, S., Hesterberg, D. \& Ravel, B., Analysis of soils and
  minerals using x-ray absorption spectroscopy, In Drees, L.\ \&
  Ulery, A.\ (eds.)  \emph{Methods of Soil Analysis - Part 5:
    Mineralogical Methods}, chap.~14, American Society of Agronomy
  (2008)
  \href{https://portal.sciencesocieties.org/Purchase/ProductDetail.aspx?Product_code=802f0511-76f0-dc11-b6b8-0013210e308c}
  {\color{color2}\homepagesymbol~Science Societies bookstore}}



\nocite{*}
\bibliographystyle{unsrt}
\bibliography{allmine}

\end{document}
